% !TeX document-id = {8c733c2e-a4ee-44a8-84dc-7d496a5f9dba}
% !TeX TXS-program:compile = txs:///lualatex
\documentclass{standalone}
\usepackage{pgfplots}
\pgfplotsset{width=7cm,compat=1.8}
\usepackage{luacode}
\usepackage{verbatim}
\begin{luacode}
  function julia(cx,cy, max_iter, max)
    local x,y,xtemp,ytemp,squaresum,iter
    squaresum = 0
    x = cx
    y = cy
    iter = 0
    while (squaresum <= max) and (iter < max_iter)  do
      xtemp = x * x - y * y - 0.742
      ytemp = 2 * x * y + 0.1
      x = xtemp
      y = ytemp
      iter = iter + 1
      squaresum = x * x + y * y
    end
    local result = 0
    if (iter < max_iter) then
        result = iter
    end
    -- result = squaresum
    -- io.write("" .. cx .. ", " .. cy .. " = " .. result .. " (iter " .. iter .. " squaresum " .. squaresum .. ") \string\n")
    tex.print(result);
  end
\end{luacode}
\begin{document}
\begin{comment}
Por favor compile com lualatex. Demora alguns minutos, seja paciente
Exemplo do site
http://pgfplots.net/tikz/examples/julia/
\end{comment}
\begin{tikzpicture}
  \begin{axis}[
    colormap/hot2,
    colorbar,
    axis equal,
    point meta max=50,
    tick label style={font=\tiny},
    view={0}{90}]
    \addplot3 [surf, domain = -1.82:1.82, shader = interp,
      domain y = -1.5:1.5, samples = 350] 
      {\directlua{julia(\pgfmathfloatvalueof\x,\pgfmathfloatvalueof\y,10000,4)}
    };
  \end{axis}
\end{tikzpicture}
\end{document}