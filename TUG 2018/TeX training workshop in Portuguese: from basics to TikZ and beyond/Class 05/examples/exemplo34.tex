% !TeX document-id = {c2f8ccfe-6fe8-4f86-9a2b-c7b7673ef5e3}
% !TeX TXS-program:compile = txs:///lualatex
\documentclass{standalone}
\usepackage{pgfplots}
\pgfplotsset{compat=1.8}
\usepackage{verbatim}
\renewcommand*{\r}{(sin(4*u)^3 + cos(2*u)^3 + sin(6*v)^2 + cos(6*v)^4)}
\begin{document}
\begin{comment}
Por favor compile com lualatex. Demora alguns minutos, seja paciente
Exemplo do site
https://texwelt.de/wissen/fragen/6952/kann-man-bei-pgfplots-die-argumente-trigonometrischer-funktionen-von-grad-auf-radiant-umstellen
\end{comment}
\begin{tikzpicture}
  \begin{axis}[
      view = {120}{310},
      hide axis,
      colormap = {flower}{
        color(0cm)  = (yellow);       color(4cm)  = (yellow!70!red);
        color(8cm)  = (red!60!black); color(12cm) = (red!90!black);
        color(16cm) = (green);        color(20cm) = (green!20!black)},
    ]
    \addplot3[
      surf,
      z buffer = sort,  point meta = u,
      domain   = 0:180, domain y   = 0:360,
      samples  = 80,    samples y  = 80,
      variable = \u,    variable y = \v
    ]
    ( { \r*sin(u)*cos(v) },
      { \r*cos(u) },
      { \r*sin(u)*sin(v) } );
  \end{axis}
\end{tikzpicture}
\end{document}