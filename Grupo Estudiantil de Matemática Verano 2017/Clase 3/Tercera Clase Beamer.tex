%
% Tercera Clase Beamer.tex -- a document for writing notes with GEM.
%
% Copyright © 2017 Luis Felipe Villavicencio<lvillavicenciol@uni.pe>
%
% This program is free software: you can redistribute it and/or modify
% it under the terms of the GNU General Public License as published by
% the Free Software Foundation, either version 3 of the License, or
% (at your option) any later version.
%
% This program is distributed in the hope that it will be useful,
% but WITHOUT ANY WARRANTY; without even the implied warranty of
% MERCHANTABILITY or FITNESS FOR A PARTICULAR PURPOSE.  See the
% GNU General Public License for more details.
%
% You should have received a copy of the GNU General Public License
%s along with this program.  If not, see <http://www.gnu.org/licenses/>.
%
\documentclass[handout]{beamer}
\usepackage[utf8x]{inputenc}
\usepackage[spanish]{babel}
\usepackage{amsmath,amsthm,amssymb,enumerate}
\author{Grupo Estudiantil de Matemática}
\title{Primera presentación en Beamer}
\institute{\LARGE{Universidad Nacional de Ingeniería\\Facultad de Ciencias}}
\date{8 de febrero del 2017} % Si deseas mostrar la fecha actual, entonces escribe \date{\today}
\newcommand{\ds}{\displaystyle}
\newcommand{\re}{\mathbb{R}}
\usetheme{Antibes}

\begin{document}

\begin{frame}
		\maketitle
\end{frame}

\begin{frame}{Esto es un título}{Esto es un subtitulo}
Esto es un frame, aquí podemos escribir fórmulas como \(\ds\int_S f(x,y)dA \).\\
También podemos crear tablas o ecuaciones centradas.

\[1+2+3+4+5+6+\cdots+n = {\color{blue}\frac{n(n+1)}{2}}.\]

{\color{red} Texto a ser coloreado de rojo.}

\begin{center}

	\begin{tabular}{||c||c||}
		\hline 1 & 2 \\ \hline
		\hline 3 & 4 \\ \hline
		\hline 
	\end{tabular}
	
\end{center}
	
\end{frame}

\begin{frame}{Tipos de Cajas}

\begin{exampleblock}{Exampleblock}
Texto o fórmulas en \LaTeX o imágenes o cuadros.
\[f(x)\subset \re.\]
\end{exampleblock}
	
\begin{alertblock}{Alertblock}
Esta caja es de color rojo.

\begin{center}
	\begin{tabular}{||c||c||}
		\hline 1 & 2 \\ \hline						
	\end{tabular} 
\end{center}
	
\end{alertblock}

\begin{block}{Block}
	Caja pequeña de color azul.
\end{block}	
\end{frame}

\begin{frame}{El poder de \LaTeX}
A continuación aparecerá una caja. \pause 
\begin{alertblock}{¡Apareció la caja!}
	Matemáticas
	\begin{itemize}
		\item Primera ítem \pause
		\item Segunda ítem  \pause
		\item Una ecuación \pause \[x^0+y^0=2,~\forall x,y\in\re\setminus\{0\}.\]
	\end{itemize}
	
\end{alertblock}
	
\end{frame}

\begin{frame}
\[1+2+3+4+5+6= \pause(1+2+3)+\pause(4+5+6) = \pause6 + \pause 15= \pause 21.\]

\pause Para eliminar los velos cambiar\\
$\backslash\!\!$ documentclass[handout] por  $\backslash\!\!$ documentclass[beamer]\\\Large{¡Gracias por tu atención!}
\end{frame}

\end{document}