% !TeX document-id = {9d223eca-9818-4f05-a953-42e014392f08}
% !TeX TXS-program:compile = txs:///pdflatex/[--shell-escape]

\documentclass{article}
\usepackage[utf8x]{inputenc}
\usepackage[spanish]{babel}
\spanishdatedel
\usepackage{amsfonts}
\usepackage[pdftex]{graphicx}
\usepackage{float}
\usepackage{vmargin}
\setpapersize{A4} % Define el tamaño del papel.
\setmargins{2.5cm} % Margen izquierdo
{1.5cm} % Margen superior
{16.5cm} % Ancho del área de impresión
{23.42cm} % Longitud del área de impresión.
{0pt} % Espacio para el encabezado
{5mm} % Espacio entre el encabezado y el texto.
{0pt} % Espacio para el pie de página.
{5mm} % Espacio entre el texto y el pie de página.
\graphicspath{./images/}
\usepackage{wrapfig}
\usepackage{lipsum}
\usepackage{subcaption}
\usepackage{slashbox} % https://ctan.org/pkg/slashbox
\usepackage{colortbl} % For the command \rowcolor{}
\usepackage[table]{xcolor} % Necessary load.
\usepackage{multicol, multirow}

\input{code.tex}
\usepackage{xcolor}

\definecolor{myGreen}{HTML}{36A736}

\definecolor{myBlue}{HTML}{02528F}

\definecolor{myOrange}{HTML}{FF4312}

\definecolor{blue254}{HTML}{02529F}

\title{Ejemplo de uso de Imágenes y Tablas}
\author{Oromion}
\date{\today}

\begin{document}
\maketitle
\renewcommand{\contentsname}{Tabla de contenido}
\renewcommand{\listfigurename}{Lista de figuras}
\renewcommand{\listtablename}{Lista de tablas}
\renewcommand{\figurename}{Fig.}
\renewcommand{\tablename}{Tabla.}
\tableofcontents
\listoffigures

\section{Construcción de tablas}

El ambiente \cmd{table} define un objeto flotante de tipo \bftt{tabla} y el símbolo \cmd{tabular} define el arreglo en filas y columnas, el uso básico del ambiente \cmd{tabular} es idéntico al uso de cualquiera de los ambientes para la definición de matrices.

\renewcommand{\tabcolsep}{10pt}
\renewcommand{\arraystretch}{1.5}
\renewcommand{\arrayrulewidth}{1pt}

\begin{wraptable}[11]{L}[5mm]{.4\textwidth}
	\centering
	\begin{tabular}{ccc}
		$\mathbf{x}$ & $\mathbf{y}$ & $\mathbf{f_{xy}(x,y)}$ \\
		\hline
		$-1$ & $-2$ & $\frac{1}{8}$ \\
		$-0.5$ & $-1$ & $\frac{1}{4}$ \\
		$0.5$ & $1$ & $\frac{1}{2}$ \\
		$1$ & $2$ & $\frac{1}{8}$ \\
		\hline 
	\end{tabular}
	\caption{Tabla con espacios automáticos}
\end{wraptable}
\lipsum[1-2]

Algunas veces dividir una celda en diagonal es algo útil, una de las formas de conseguirlo es utilizar el comando \cmdbs{backslashbox} del paquete \cmd{slashbox}, este no es un paquete estándar.

\begin{table}[H]
	\centering
	\begin{tabular}{|l||c|c|c|}
		\hline
		\backslashbox{Adición}{Cesión} & CNC & CNS & CM \\
		\hline \hline
		CNC & & & \\
		\hline
		CNS & & & \\
		\hline
		CM & & & \\
		\hline
	\end{tabular}
	\caption{Tabla con una celda dividida en diagonal}
\end{table}

\subsection{Agregando color a una tabla}

Para agregar colores vamos a utilizar el paquete \cmd{colortbl}.

\begin{table}[H]
	\centering
	\begin{tabular}{|c|c|>{\columncolor{emeraldGreen!50}}c|>{\columncolor{myOrange!30}}c|}
		\hline
		\rowcolor{signBlue!50}
		\textbf{Clase} & $\mathbf{x_i}$ & $\mathbf{f_i}$ & $\mathbf{h_i}$\\
		\hline
		$[5, 10]$ & 7.5 & 5 & 0.5 \\
		\hline
		$[10, 15]$ & 12.5 & 8 & 0.24 \\
		\hline
		$[15, 20]$ & 17.5 & 8 & 0.24 \\
		\hline
		$[20, 25]$ & \cellcolor{myRed!75}{22.5} & 10 & 0.39 \\
		\hline
		$[25, 30]$ & 27.5 & 2 & 0.07 \\
		\hline
	\end{tabular}
	\caption{Tabla con colores}
\end{table}

\lipsum[1-2]

\begin{table}[H]
	\centering
	\caption{Ejemplo de colores en tablas}
	\rowcolors{1}{darkOliveGreen2!35}{myBlue!35}
	\begin{tabular}{lc}
		\hline
		Población empadronada en España & 46.600.949 \\
		Población española & 41.882.085 \\
		Población extranjera & 4.718.864 {10.1\%} \\
		Población extranjera de 16 años & 16\% \\
		Población extranjera $<$ 16 años & 15.8 \% \\
		Países de procedencia más frecuentes & \\
		\begin{tabular}{ll}
			Rumania & 15.9\% \\
			Marruecos & 15.8 \% \\
			China & 4.05 \% \\
		\end{tabular} & \\
		\hline
	\end{tabular}
\end{table}

\section{Combinar celdas}

\begin{table}[H]
	\centering
	\begin{tabular}{|>{\cellcolor{myBlue!75}}cc|c|c|}
		\hline
		\rowcolor{myBlue!75}
		& \multicolumn{3}{c|}{\textcolor{white}{Tolerancia Resistiva ($\pm$)}} \\
		\rowcolor{myBlue!75}
		& \textcolor{white}{20\%} & \textcolor{white}{10\%} & \textcolor{white}{5\%} \\
		& \multirow{4}{*}{100} & \multirow{2}{*}{100} & 100 \\
		& & & 91 \\
		\cline{3-4}
		& & \multirow{2}{*}{82} & 82\\
		& & & 75 \\
		\cline{2-4}
		& \multirow{4}{*}{68} & \multirow{2}{*}{68} & 68 \\
		& & & 62 \\
		\cline{3-4}
		& & \multirow{2}{*}{56} & 56 \\
		& & & 51 \\
		\cline{2-4}
		& \multirow{4}{*}{47} & \multirow{2}{*}{47} & 47 \\
		& & & 43 \\
		\cline{3-4}
		& & \multirow{2}{*}{39} & 39 \\
		& & & 36 \\
		\cline{2-4}
		& \multirow{4}{*}{33} & \multirow{2}{*}{33} & 33 \\
		& & & 30 \\
		\cline{3-4}
		& & \multirow{2}{*}{27} & 27 \\
		& & & 24 \\
		\cline{2-4}
		& \multirow{4}{*}{22} & \multirow{2}{*}{22} & 22 \\
		& & & 20 \\
		\cline{3-4}
		& & \multirow{2}{*}{18} & 18 \\
		& & & 16 \\
		\cline{2-4}
		& \multirow{4}{*}{15} & \multirow{2}{*}{15} & 15 \\
		& & & 13 \\
		\cline{3-4}
		& & \multirow{2}{*}{12} & 12 \\
		& & & 11 \\
		\cline{2-4}
		\multirow{-25}{*}{\rotatebox[origin=c]{90}{\textcolor{white}{Valores de Resistencia Estándar}}} & 10 & 10 & 10 \\
		\hline
	\end{tabular}
\end{table}
\end{document}