\documentclass{article}
\usepackage[utf8x]{inputenc}
\usepackage[spanish]{babel}
\spanishdatedel
\usepackage[pdftex]{graphicx}
\usepackage{float}
\usepackage{subcaption}
\usepackage{vmargin}
\setpapersize{A4} % Define el tamaño del papel.
\setmargins{2.5cm} % Margen izquierdo
{1.5cm} % Margen superior
{16.5cm} % Ancho del área de impresión
{23.42cm} % Longitud del área de impresión.
{0pt} % Espacio para el encabezado
{5mm} % Espacio entre el encabezado y el texto.
{0pt} % Espacio para el pie de página.
{5mm} % Espacio entre el texto y el pie de página.
\graphicspath{./images/}
\usepackage{wrapfig}
\usepackage{lipsum}

\input{code.tex}

\title{Ejemplo de uso de Imágenes y Tablas}
\author{Oromion}
\date{\today}
\includegraphics[scale=1]{gemtransparente.}

\begin{document}
\maketitle
\renewcommand{\contentsname}{Tabla de contenido}
\renewcommand{\listfigurename}{Lista de figuras}
\renewcommand{\figurename}{Fig.}
\tableofcontents
\listoffigures

\section{Incluir imágenes y figuras}

El ambiente \cmd{figure} permite definir un objeto flotante que corresponde a imágenes.

\begin{figure}[H]
	\centering
	\includegraphics[scale=.45]{example-image-a}
	\caption{Flujo de temperatura en un cubo}
\end{figure}

\subsection{Utilizar el ambiente \bftt{wrapfigure}}

\begin{wrapfigure}[11]{R}[5mm]{.45\textwidth}
	\centering
	\caption{Flujo de temperatura en un cilindro}
	\includegraphics[scale=.25]{example-image-b}
\end{wrapfigure}
\lipsum[1-2]

\subsection{Utilizar el paquete \bftt{subcaption}}

El paquete \cmd{subcaption} permite ubicar subfiguras cada una con su respectivo caption, dentro de un solo ambiente \cmd{figure}.

\begin{figure}[H]
	\centering
	\begin{subfigure}[t]{.475\textwidth}
		\centering
		\includegraphics[scale=.45]{example-image-a}
		\caption{Dispersión de la gripe}
	\end{subfigure}
	\hfill
	\begin{subfigure}[t]{.475\textwidth}
		\centering
		\includegraphics[scale=.35]{example-image-b}
		\caption{Clasificación de vértices de acuerdo al grado}
	\end{subfigure}
	\caption{Subfiguras con el paquete \bftt{subcaption}}
\end{figure}

\end{document}