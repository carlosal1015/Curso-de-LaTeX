\documentclass{article}
\usepackage[utf8x]{inputenc}
\usepackage[spanish]{babel}
\spanishdatedel
\usepackage{lipsum}

\newtheorem{definicion}{Definición}
\newtheorem{obs}{Observación}[section] % Numeración continua en cada sección.
\input{cajas.tex}
\usepackage{xcolor}

\definecolor{myGreen}{HTML}{36A736}

\definecolor{myBlue}{HTML}{02528F}

\definecolor{myOrange}{HTML}{FF4312}

\definecolor{blue254}{HTML}{02529F}
\title{Tomando el control de \LaTeX{}}
\author{Oromion}
\date{\today}
\begin{document}
\maketitle

\section{Sección 1}

\lipsum[1]

\begin{definicion}
\lipsum[1]
\end{definicion}

\begin{obs}
Esta es una primera observación de la sección 1.
\end{obs}

\lipsum[1]

\begin{definicion}
\lipsum[1]
\end{definicion}


\begin{obs}
Esta es una segunda observación de la sección 1.
\end{obs}

\section{Sección 2}

\begin{obs}
Primera observación de la sección 2.
\end{obs}

\begin{remark}{Nota importante}
	\lipsum[1]
\end{remark}

\end{document}
https://github.com/piratax007/LaTeX_Course/tree/2ad16f1eeef80f35ab7fdabca01b980f6a51a7c5/ejemplos/leccion_4