\documentclass{article}
\usepackage[utf8x]{inputenc}
\usepackage[spanish]{babel}
\spanishdatedel
\usepackage{amsmath}
\usepackage{lipsum}
\usepackage{multicol}
\usepackage{paracol}%Paquete compatible con multicol.
\usepackage{xcolor}
\title{Título del documento}
\author{Oromion}
\date{\today}

\begin{document}
	\maketitle

	\begin{abstract}
		\lipsum[2]
	\end{abstract}
	%\twocolumn[a] Aquí no se puede insertar columnas.
	%\lipsum[1]
	%\renewcommand{\columnseprule}{.5pt}
	%\setlength\columnsep{10mm}
	%\begin{multicols}{3}
	%\lipsum[2-4]
	%\end{multicols}
	\begin{paracol}{2}[\section{Sección de título largo}]
		\lipsum[2]
		\switchcolumn
		Esta es la traducción del texto anterior.
		\switchcolumn*
		\lipsum[3]
		\switchcolumn
		\lipsum[2]
	\end{paracol}
	\subsection{Columnas dispares}
	\columnratio{.6,0.3}
	\begin{paracol}{3}
		\lipsum[5]%60\%
		\switchcolumn
		\lipsum[6]%40\%
		\switchcolumn
		\lipsum[7]
	\end{paracol}
%\section*{Verdadera sección 1}
%\lipsum[5]
%\subsection{Subtitulo a .....}
%\lipsum[3]

%\subsubsection{Una sección más}

%\section{Una sección más}
%\lipsum[1]

%\section{Título de la sección 1}
%\lipsum[1-3]
%\paragraph{Otro título}
%\lipsum[2]
\end{document}