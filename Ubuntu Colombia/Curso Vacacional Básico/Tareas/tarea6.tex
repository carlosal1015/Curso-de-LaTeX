% tarea6.tex
%
% Copyright © 2018 Oromion <caznaranl@uni.pe>
%
% This program is free software: you can redistribute it and/or modify
% it under the terms of the GNU General Public License as published by
% the Free Software Foundation, either version 3 of the License, or
% (at your option) any later version.
%
% This program is distributed in the hope that it will be useful,
% but WITHOUT ANY WARRANTY; without even the implied warranty of
% MERCHANTABILITY or FITNESS FOR A PARTICULAR PURPOSE.  See the
% GNU General Public License for more details.
%
% You should have received a copy of the GNU General Public License
% along with this program.  If not, see <http://www.gnu.org/licenses/>.
%
\documentclass[12pt]{beamer}
\setbeamercolor{title}{bg = magenta!50!black, fg= white}
\usepackage[utf8]{inputenc}
\usepackage[spanish]{babel}
\spanishdatedel
\uselanguage{Spanish} % Teoremas en idioma que elijas.
\languagepath{Spanish}
\input{code.tex}
% Lista de \newcommand{cmd}[args]{def}

\newcommand{\gammita}{\Gamma(x)=\int_{0}^{\infty}e^{-t}\,t^{x-1}\mathrm{dt}}
\newcommand{\polyrational}[2]{\dfrac{#1}{#2}}
\newcommand{\Rsqr}[1][]{\ensuremath{\mathrm{R_{#1}^2}}}
\newcommand{\vect}[1]{(#1_1,#1_2,\dots,#1_n)}

% Lista de \renewcommand{cmd}[args]{def}
\renewcommand{\gammita}{\frac{\partial^2 u}{\partial t^2} = c^2 \frac{\partial^2 u}{\partial x^2}}

\usepackage{cprotect} % Allow \cprotect for display code with \verb|code|
\usepackage{graphicx}
\usepackage{verbatim}
\usepackage{enumerate}
\usetheme{Frankfurt}
\usecolortheme{dolphin}
\usefonttheme{professionalfonts}

\begin{document}
\author{Matemático Oromion}
\title{\Large Renombrando comandos en \LaTeX{} con}
\cprotect\subtitle{\verb|\renewcommand{cmd}[args]{def}|}
\logo{\includegraphics[scale=.5]{peru.jpg}}
\institute{\large Universidad del Perú}
\date{\textcolor{magenta}{\today}}
\subject{Fin del curso}
\setbeamercovered{transparent}
\setbeamertemplate{navigation symbols}{}

\begin{frame}[plain]
	\maketitle
\end{frame}

\begin{frame}{Contenidos}{Presentación final}
\tableofcontents
\end{frame}

\section{Introducción}
\cprotect\subsection{Entiendiendo el comando: \verb!\newcommand{\newbox}[args]{def}!}

\begin{frame}[fragile]
\frametitle{\thesection.\thesubsection. Entendiendo el comando \cmd{newcommand} \footnote{Por favor, vea la lista completa de comandos en \texttt{comandos.tex} }}
\centering
\verb|\newcommand{cmd}[args]{def}|

\begin{columns}
\column{0.5\textwidth}


\begin{itemize}
	\item El primer argumento obligatorio es para nombrar el \cmd{comando}.
	\item El primer argumento opcional indica la cantidad de parámetros.
	\item El segundo argumento obligatorio recibe la acción.
\end{itemize}
\column{0.5\textwidth}
\begin{block}{Ejemplos}
\begin{enumerate}[a)]
\item Con un argumento.
$\gammita\quad\Rsqr[1]$
\item Con dos argumentos.
\[\polyrational{ax^{2}+bx+c}{x^{p}-1}\]
\end{enumerate}
\end{block}

\begin{alertblock}{¡Cuidado!}
Se comete un error si el comando ya está definido.
\end{alertblock}
\end{columns}

\end{frame}

\cprotect\subsection{Aprendiendo el comando: \verb!\renewcommand{cmd}[args]{def}!}

\begin{frame}[fragile]
\frametitle{\thesection.\thesubsection. Aprendiendo el comando \cmd{renewcommand}}
\vspace*{-2cm}
\centering
\verb|\renewcommand{cmd}[args]{def}|

\begin{block}{Comando \cmd{renewcommand}}
Redefine un comando predefinido y comete un error si aún no está definido.
\end{block}
\end{frame}

\cprotect\subsection{Extra: \verb!\providecommand{cmd}[args]{def}!}

\begin{frame}[fragile]
\frametitle{\thesection.\thesubsection. Extra: \cmd{providecommand}}
\vspace*{-2cm}
\centering
\verb|\providecommand{cmd}[args]{def}|
\begin{alertblock}{¡Cuidado!}
Define un nuevo comando si aún no está definido.
\end{alertblock}
\end{frame}

\cprotect\subsection{Usando el comando robusto: \verb!\@!}

\section{Ejemplos}

\subsection{Definiendo el estilo del modo matemático de una }

\begin{frame}
\frametitle{\thesection. \thesubsection. Ecuación en derivadas parciales}
Comando renombrado \cmd{gammita}
\[\gammita\]
\end{frame}

\subsection{Creando un comando de una caja}

\section{\thesection. Agradecimientos}
\subsection{Ubuntu Colombia}

\begin{frame}
\frametitle{Agradecimientos}
\Large
Felicito y agradezco al profesor Fausto Suárez y a la coordinadora Lina Porras por su apoyo a la difusión de \LaTeX{} en Sudamérica.
\end{frame}
\end{document}