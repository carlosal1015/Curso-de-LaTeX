%Tercer ejercicio.
\documentclass{standalone}
\usepackage[utf8]{inputenc}
\usepackage[spanish]{babel}
\usepackage[T1]{fontenc}
\usepackage{PTSansNarrow}
\usepackage[usenames,dvipsnames,x11names,table,svgnames]{xcolor}
\usepackage{tikz}
\usetikzlibrary{calc,babel,through,intersections,backgrounds}
\begin{document}
\tikzset{help lines/.style=very thin}
\begin{tikzpicture}[scale=2]
\clip (-0.1,-0.2) rectangle (1.1,0.75);
\draw[help lines] (0,0) grid (5,5);
\draw (-1.5,0) -- (1.5,0);
\draw (0,-1.5) -- (0,1.5);
\draw (-1,0) .. controls (-1,0.555) and (-0.555,1) .. (0,1) .. controls (0.555,1) and (1,0.555) .. (1,0);
\draw[rotate=30] (0,0) ellipse [x radius=6pt, y radius=3pt];
\draw (2.5,2.5) circle [radius=2.5cm];
\draw[x=1pt,y=1pt] (0,0) parabola bend (4,16) (6,12);
\draw[x=1.57ex,y=1ex] (0,0) sin (1,1) cos (2,0) sin (3,-1) cos (4,0)
					(0,1) cos (1,0) sin (2,-1) cos (3,0) sin (4,1);
%\fill[green!20!white] (0,0) -- (3mm,0mm)
%arc [start angle=0, end angle=30, radius=3mm] -- (0,0);
\fill[green!20!white] (0,0) -- (3mm,0mm)
arc [start angle=0, end angle=30, radius=3mm] -- cycle;
%cycle cierra la curva de Jordan.
\end{tikzpicture}
\end{document}