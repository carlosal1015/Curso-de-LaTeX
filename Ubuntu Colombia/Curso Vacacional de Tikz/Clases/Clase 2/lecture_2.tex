\documentclass{standalone}
\usepackage[utf8]{inputenc}
\usepackage[spanish]{babel}

\usepackage{tikz}
\usetikzlibrary{calc,babel,through,intersections,backgrounds}
\begin{document}
	\begin{tikzpicture}
	\coordinate[label = left:{\textcolor{blue}{$A$}}] (A) at (0, 0);
	%Establece las coordenadas, lo rotula.
	%No necesariamente debe coincidir el nombre.
	\coordinate[label = right:{\textcolor{blue}{$B$}}] (B) at (1.15, .75);
	
	\draw[blue] (A) -- (B);
	
	\node (D) [name path = D, draw, circle through = (B), label = left: $D$]  at (A) {};
	%Etiqueta es diferente a nombre. El nodo se llama B, pero me refiero al punto A (sus coordenadas). El corchete debe acabar, y puede ser vacío.
	\node (E) [name path = E, draw, circle through = (A), label = right: $E$]  at (B) {};
	%name path funciona con intersections y lo vuelve un camino.
	
	\path[name intersections = {of = D and E}];
	%Lo que se ha hecho es hallar las intersecciones y las enumera.
	
	\coordinate[label = above:{\textcolor{red}{$C$}}] (C) at (intersection-1);
	%\coordinate[label = above:{\textcolor{red}{$C$}}] (C) at (intersection-2); %esta es la intersección inferior.
	
	%\draw[red] (A) -- (C);
	%\draw[red] (B) -- (C);
	%Es equivalente a la proposición anterior.
	
	\draw[red] (A) -- (C) -- (B);
	
	%\fill[black, opacity = .5] (A) circle[radius =1pt];
	%\fill[black, opacity = .5] (B) circle[radius =1pt];
	%\fill[black, opacity = .5] (C) circle[radius =1pt];
	
	\foreach \point in {A, B, C}{
		\fill[black, opacity = .5] (\point) circle[radius = 1.5pt];
	}
	%Hemos visto el bloque for each.
	
	%\fill[orange] (A) -- (B) -- (C) -- cycle;
	%Tikz es un lenguaje secuencial. No se puede colocar antes de los puntos A, B y C.
	\begin{pgfonlayer}{background}
		\fill[orange] (A) -- (B) -- (C) -- cycle;
	\end{pgfonlayer}
	
	\node[below, text width = 10cm, align = justify] at (.5,-1.5)
{
	\small\textbf{Proposición 1:}\par
	\emph{Construir un \textcolor{orange}{triángulo equilátero} sobre un
	\textcolor{blue}{segmento de recta} dado.}
	\par\vskip2em
	Sea \textcolor{blue}{AB} un segmento de recta,
	contruyendo...
	};
	\end{tikzpicture}
	\end{document}
La librería intersections sirve para identificar intersecciones, la librería backgrounds sirve para crear capas (en Inkscape también se puede utilizar).	

through sirve para poder dibujar un círculo y un punto que pase por allí.

