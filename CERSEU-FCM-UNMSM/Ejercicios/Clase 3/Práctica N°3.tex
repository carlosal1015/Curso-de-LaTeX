\documentclass[12pt,a4paper]{article}
\usepackage[utf8]{inputenc}
\usepackage[spanish]{babel}
\usepackage{amsmath}
\usepackage{amsfonts}
\usepackage{amssymb}
\usepackage{graphicx}
\usepackage[dvipsnames,svgnames]{xcolor}
\usepackage{fancyhdr}
\lhead{UNMSM-FCM-Ceups}
\chead{Curso de \LaTeX}
\rhead{Práctica dirigida N$^{\circ}$2}
\lfoot{\textit{Prof. Carlos Alonso Aznarán Laos}}
\cfoot{\thepage}
\rfoot{\today}
\pagestyle{fancy}
\usepackage[left=2cm,right=2cm,top=3cm,bottom=3cm]{geometry}
\author{\textit{Carlos Alonso Aznarán Laos}}
\title{{Curso de \LaTeX}\\{\normalsize \textbf{{Práctica N$ ^{\circ}$3}}}%cómo colorear la fecha
\begin{document}
\maketitle
%\noident
%{\color{Red} \rule{\linewidth}{0.5mm}
\begin{abstract}
La práctica dirigida N$ ^{\circ}$3 tiene como objetivo ejercitar a los asistentes al curso de \LaTeX en el uso de formatos de edición de un documento. Comprende : formatos de párrafo, de letras, de página y la elaboración de listas.
\end{abstract}

\section{Formatos}

\subsection{Alinear y enfatizar texto}
\LaTeX presenta por defecto, textos justificados. Pero podemos cambiar la alineación de la siguiente manera:
\begin{center}
\textcolor{Blue}{Texto \textbf{\underline{centrado}}}
\end{center}
\begin{flushright}
\textcolor{Blue}{
\textit{Texto alineado a la \underline{\textbf{derecha}}}}
\end{flushright}
En este ejercicio también se \textcolor{Red}{\underline{practicó}} cómo \textcolor{Red}{\underline{enfatizar}} textos.

\section{Tipo de letra y entornos quote y quotation}

Comando \textbf{quote} y tipo \textbf{sffamily}.
\begin{quote}
\sffamily
\textcolor{Blue}{<<\ldots si la anomia es un mal, lo es, ante todo, porque la sociedad lo sufre, no pudiendo prescindir, para vivir, de cohesión y regularidad>>. Emilio Durkheim}
\end{quote}
Comando \textbf{quote} y tipo \textbf{ttfamily}.
\begin{quote}
\ttfamily
<<\ldots si la anomia es un mal, lo es, ante todo, porque la sociedad lo sufre, no pudiendo prescindir, para vivir, de cohesión y regularidad.>> Emilio Durkheim
\end{quote}

\section{Algunos símbolos especiales}

\noindent El precio del barril de petróleo el 14 de agosto de 2015: US\$ 42.74.\\
El conjunto $a=\{2,4,6,8\}$\\
Descuento por temporada: 50\%\\
Clásico de la sociología de la educación: \textit{La  reproducción} (Bordieu \& Passeron,1970).
\pagebreak

\section{Formatos de página}

\pagecolor{LimeGreen}
Aplicar el color SpringGreen a esta página.

\section{Escribir listas}

En realidad, no existe un único formato único de presentación de artículos científicos. Sin embargo las revistas científicas, tienden a solicitar a sus contribuyentes que incluyan las siguientes secciones:
\begin{enumerate}
\item Resumen
\item Introducción
\item Antecedentes
\begin{itemize}
\item[a)] Contexto histórico
\item[b)] Teoría
\begin{description}
\item[Teorías a favor.] Estas teorías apoyan la hipótesis aceptada por el investigador.
\item[Teorías en contra.] Estas teorías cuestionan la hipótesis.
\end{description}
\item[c)] Revisión de la literatura
\end{itemize}
\item Datos y metodogía
\item Resultados
\item Discusión
\item Conclusiones
\item Bibliografía
\end{enumerate}
\begin{thebibliography}{9} \bibitem{Hellus2015} Hellus, M.; Waldi, R. On the number of numerical semigroups containing two coprime integers $p$ and $q$. Semigroup Forum 90 (2015), no. 3, 833--842.
\bibitem{venables} Venables, W. N., Smith D. M.and the R Core Team. An Introduction to R CRAN (2015).
\bibitem{griffith} Griffith, A. B., Salguero-Gómez, R., Merow, C., and McMahon, S. (2016). Demography beyond the population. Journal of Ecology, 104(2), 271-280.
\end{thebibliography}

\appendix

\section{Código \LaTeX}

\begin{verbatim}
\documentclass[12pt,a4paper]{article}
\author{Carlos Alonso Aznarán Laos}
\begin{document}
\maketitle
\end{document}
\end{verbatim}
\end{document}