\documentclass[12pt,a4paper]{article}
\usepackage[utf8]{inputenc}
\usepackage[spanish]{babel}
\usepackage{amsmath}
\usepackage{amsfonts}
\usepackage{amssymb}
%\usepackage{undertilde}
\usepackage{amsthm}
\usepackage{graphicx}
\usepackage{fancyhdr}
\usepackage[left=3cm,right=3cm,top=2.5cm,bottom=2.5cm]{geometry}
\author{Registrar nombres y apellidos}
\title{Curso de \LaTeX{}\\{\normalsize Práctica N$^{\circ}$ 2: escribir matemáticas}}
\date{}
\lhead{UNMSM-FCM-Cerseu}
\chead{}
\rhead{Curso de \LaTeX}
\lfoot{Prof. R. Collatón Ch.}
\cfoot{\thepage}
\rfoot{\today}
\pagestyle{fancy}
\setlength{\headheight}{15pt}
\newtheorem{ejer}{Ejercicio}
\begin{document}
\maketitle

\begin{ejer}
Raíces, fracciones y exponentes
\begin{displaymath}
\sqrt[n]{\frac{a}{b}}= \left(\frac{a}{b}\right)^{\frac{1}{n}}=\frac{a^{\frac{1}{n}}}{b^{\frac{1}{n}}}=\frac{\sqrt[n]{a}}{\sqrt[n]{b}}
\end{displaymath}
\end{ejer}

\begin{ejer}
Uso del entorno gather
\begin{gather}
x^{y^{z}}=a\\
x^{2}_{1}=x^{2}_{1}\\
x^{2y}=z\\
a_{ij}=0
\end{gather}
\end{ejer}

\begin{ejer}
Letras griegas, sombreros y sumatoria

\begin{displaymath}
n^{-1}\sum^{n}_{i=1}(y_{i}- \hat{\beta}_{0}-\hat{\beta}_{1}x_{i})=0
\end{displaymath}

\begin{displaymath}
\bar{y}=\hat{\beta}_{0}+\hat{\beta}_{1}\bar{x},
\end{displaymath}
\end{ejer}
donde $\bar{y}=n^{-1}\sum^{n}_{i=1}y_{i}$ es la muestra promedio de $y_{i}$ e igualmente por $\bar{x}$. 
%\end{ejer}

\begin{ejer}
Ecuaciones: uso de entorno equation y entorno cases
\begin{equation*}
|x| =
\begin{cases}
a & \text{if $a\ge 0$}\\
-a & \text{if $a< 0$}
\end{cases}
\end{equation*}
\end{ejer}
\begin{ejer}
Uso de entorno equation y entorno split
\begin{equation}
\begin{split}
a& =b+c-d\\
& \quad +e-f\\
& =g+h\\
& =i
\end{split}
\end{equation}
\end{ejer}
\begin{ejer}
Uso del entorno gather
\begin{gather}
a_1=b_1+c_1\\
a_2=b_2+c_2-d_2+e_2
\end{gather}
\end{ejer}
\begin{ejer}
Uso del entorno align
\begin{align}
a_1& =b_1+c_1\\
a_2& =b_2+c_2-d_2+e_2
\end{align}
\end{ejer}


\begin{ejer}
Matrices

Dado que
$
\left|\begin{smallmatrix}
a & h & g\\
h & b & f\\
g & f & c
\end{smallmatrix}\right|
=0
$,
la matriz
$
\left(\begin{smallmatrix}
a & h & g\\
h & b & f\\
g & f & c
\end{smallmatrix}\right)
$
no es invertible.
\end{ejer}


\begin{ejer}
\[
\overbrace{x+\underbrace{y+z}_{2} +w}^{4} 
\]

\begin{displaymath}
a^{n}=\overbrace{a \times a \times \ldots \times a}^{\mbox{$n$ veces}}
\end{displaymath}
\end{ejer}






\nocite{*}

\begin{thebibliography}{9}
\bibitem{mat}
Harshbarger, R., \& Reynolds, J. J. (2013). Mathematical applications for the management, life, and social sciences. Cengage Learning.

\bibitem{harald}
Helfgott, H. (2013). La conjetura débil de Goldbach. La Gaceta de la RSME, 16(4), 709-726.

\bibitem{eco}
Wooldridge, J. M. (2015). Introductory econometrics: A modern approach. Cengage Learning.
\end{thebibliography}
\end{document}
