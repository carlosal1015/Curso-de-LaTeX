\section{Inclusión de figuras en \LaTeX}

\begin{frame}[fragile]
	\frametitle{\secname}

	Al utilizar los paquetes gráficos de \LaTeX, el espacio necesario
	para el material tipográfico tras la inclusión de un archivo o la
	aplicación de una transformación geométrica se reserva en la página
	de salida.
	Sin embargo, es responsabilidad del controlador del dispositivo
	(p. ej., \mintinline{latex}|dvips|, \mintinline{latex}|xdvi|,
	\mintinline{latex}|dvipsone|) realizar la inclusión o
	transformación en cuestión y mostrar el resultado correcto.
	La sintaxis del entorno \mintinline{latex}|figure| es el siguiente
	\inputminted[fontsize=\tiny,frame=single]{latex}{sections/figure.tex}
	donde la lista llave=valor es una lista separada por comas de
	pares \mintinline{latex}|llave=valor| para claves que toman un
	valor.
	\begin{columns}
		\begin{column}{.48\paperwidth}
			\begin{itemize}
				\item

				      \mintinline{latex}|[bb={1pt,2pt,3pt,4pt}]|.
				      El cuadro delimitador de la imagen.

				\item

				      \mintinline{latex}|[angle=2]|.
				      El ángulo de rotación (en grados, en sentido
				      antihorario).

				\item

				      \mintinline{latex}|[width=3cm]|.
				      El ancho requerido (el ancho de la imagen se escala a
				      ese valor).
			\end{itemize}
		\end{column}
		\begin{column}{.48\paperwidth}
			\begin{itemize}
				\item

				      \mintinline{latex}|[height=4cm]|.
				      La altura requerida (la altura de la imagen se escala a
				      ese valor).

				\item

				      \mintinline{latex}|[scale=0.5]|.
				      El factor de escala.

				\item

				      \mintinline{latex}|[clip=true]|.
				      Recorta el gráfico al cuadro delimitador o al
				      rectángulo especificado por la llave
				      \mintinline{latex}|trim|.
			\end{itemize}
		\end{column}
	\end{columns}

	\begin{example}% tex-fmt: off
\begin{exampletwoup}
%\usepackage{graphicx}
\begin{figure}[ht!]
\centering
\includegraphics[width=.2\linewidth,angle=9]{logouni}
\includegraphics[width=.2\linewidth,angle=-9]{gem}
\caption{Logos.}
\end{figure}
\end{exampletwoup}
% tex-fmt: on
\end{example} % tex-fmt: skip
\end{frame}
