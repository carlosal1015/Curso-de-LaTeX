\section{Unidades de medida}

\begin{frame}
	\frametitle{\secname}

	\begin{block}{Unidades de medida y longitudes en \LaTeX}
		\LaTeX{} reconoce solamente las unidades de medida básicas de \TeX.
		\begin{table}[ht!]
			\centering
			\begin{tabular}{ccl}
				\hline
				Unidades de \TeX & Abreviatura            & Significado                                                                              \\
				\hline
				Pulgadas         & \mintinline{latex}|in| & Usual                                                                                    \\
				\hline
				Centímetros      & \mintinline{latex}|cm| & 1\mintinline{latex}|cm|=28.5\mintinline{latex}|pt|                                       \\
				\hline
				Milímetros       & \mintinline{latex}|mm| & Usual                                                                                    \\
				\hline
				Puntos           & \mintinline{latex}|pt| & 1\mintinline{latex}|pt|=ancho de un punto$\approx 0.35146$\mintinline{latex}|cm|         \\
				\hline
				Picas            & \mintinline{latex}|pc| & 1\mintinline{latex}|pc|=12 puntos                                                        \\
				\hline
				Emes             & \mintinline{latex}|em| & 1\mintinline{latex}|em|=ancho de una \mintinline{latex}|M| en el tamaño de letra vigente \\
				\hline
				Equis            & \mintinline{latex}|ex| & 1\mintinline{latex}|ex|=ancho de una \mintinline{latex}|x| en el tamaño de letra vigente \\
				\hline
			\end{tabular}
			\caption{Unidades de medida en \LaTeX.}
		\end{table}
	\end{block}

	\pause

	\begin{block}{Espaciado vertical}
		Para añadir espacio vertical de una \mintinline{latex}|<longitud-con-unidad>|
		determinada podemos usar las siguientes instrucciones.
		\vspace*{-\baselineskip}\setlength\belowdisplayshortskip{0pt}
		\begin{itemize}
			\item

			      \mintinline{latex}|\\[<longitud-con-unidade>]|:
			      \LaTeX{} inserta un nuevo renglón con un espaciado
			      vertical de la \mintinline{latex}|<longitud-con-unidade>|
			      dada.

			\item

			      \mintinline{latex}|\vspace{<longitud-con-unidad>}|:
			      \LaTeX{} inserta un espaciado vertical de
			      \mintinline{latex}|<longitud-con-unidade>|.

			\item

			      \mintinline{latex}|\newline|: Tiene el mismo efecto de
			      \mintinline{latex}|\\|.

			\item

			      \mintinline{latex}|\linebreak|: Justifica el renglón
			      actual, es decir, estira proporcionalmente todos los
			      caracteres hasta tocar el margen derecho y comienza un
			      nuevo renglón, sin sangría.
		\end{itemize}
	\end{block}
\end{frame}

\begin{frame}
	\frametitle{}

	\begin{block}{Control sobre cambios de página}
		\begin{itemize}
			\item

			      \mintinline{latex}|\newpage|: Inserta una nueva página.

			\item

			      \mintinline{latex}|\pagebreak|:

			      Justifica verticalmente el contenido de la página
			      insertando espacio adicional entre los párrafos
			      (no entre los renglones) y comienza una nueva página.

			\item

			      \mintinline{latex}|\clearpage|:

			      Es similar a \mintinline{latex}|\newpage| excepto que las
			      tablas o figuras que estén bajo el alcance de los
			      entornos \mintinline{latex}|table| o
			      \mintinline{latex}|figure|, y que no hayan sido colocadas
			      por \LaTeX{}, se imprimen en una o más hojas separadas.

			\item

			      \mintinline{latex}|\cleardoublepage|:
			      Funciona como \mintinline{latex}|\clearpage| para
			      documentos con la opción \mintinline{latex}|twoside|.
			      \LaTeX{} añade toda una hoja en blanco adicional, si es
			      necesario, para que la siguiente página de texto tenga
			      numeración impar.
		\end{itemize}
	\end{block}

	\begin{block}{Alinación de texto}% tex-fmt: off
\begin{exampletwoup}
\centerline{Lee esta frase, por favor.}

\begin{center}
``El sentido común es la cosa mejor repartida del \\
mundo, ya que cada uno piensa estar tan bien \\
provisto de él, que incluso los que son difíciles \\
de contentar no suelen desear más del que poseen''.
Descartes
\end{center}

\begin{flushright}
Si quieres que el futuro sea diferente\\
del presente debes conocer el pasado.\\
Baruch Spinoza (1632--1677)
\end{flushright}
\end{exampletwoup}
% tex-fmt: on
\end{block} % tex-fmt: skip
\end{frame}

\begin{frame}
	\begin{block}{Cajas de texto}\mbox{text}

\fbox{text}

\makebox[1cm][r]{text}

\parbox{2cm}{text}
\footnote{texto de la nota}
\end{block} % tex-fmt: skip
\end{frame}
% siunitx enumitem
