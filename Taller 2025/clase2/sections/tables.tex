\section{Generación de tablas en \LaTeX}

\begin{frame}[fragile]
	\frametitle{\secname}

	La sintaxis del entorno \mintinline{latex}|tabular| es el siguiente
	\inputminted[fontsize=\tiny,frame=single]{latex}{sections/tabular}
	donde uno de los argumentos válidos para la definición de la
	columna son
	\begin{columns}
		\begin{column}{0.48\textwidth}
			\begin{itemize}
				\item

				      \verb|l|: la columna es alineada a la izquierda.

				\item

				      \verb|c|: la columna es alineada al centro.
			\end{itemize}
		\end{column}
		\begin{column}{0.48\textwidth}
			\begin{itemize}
				\item

				      \verb|r|: la columna es alineada a la derecha.

				\item

				      \verb|||: línea vertical.
			\end{itemize}
		\end{column}
	\end{columns}
	\begin{itemize}
		\item

		      \verb|p{longitud}|: corresponde a la definición
		      \verb|\parbox[c]{longitud}|, el texto está justificado.
	\end{itemize}
	Para las líneas horizontales se tiene los comandos
	\mintinline{latex}|\hline| o por rango
	\mintinline{latex}|\cline{inicio-fin}|.

	El comando
	\mintinline{latex}|\multicolumn{número de columnas}{definición de la columna}{contenido}|
	permite fusionar las celdas en una misma fila.

	\begin{example}% tex-fmt: off
\begin{exampletwoup}
\begin{table}[ht!]
\centering
\caption{Una tabla con \LaTeX.}
\begin{tabular}{|l|c|p{2cm}|r|}
\hline
1                          & 2       & 3 & 4   \\
\cline{2-3}
$\varphi$                  & $\zeta$ &   & $0$ \\
\hline
\multicolumn{3}{|c|}{\TeX} & \LaTeX            \\
\hline
\end{tabular}
\end{table}
\end{exampletwoup}
% tex-fmt: on
\end{example} % tex-fmt: skip
\end{frame}
