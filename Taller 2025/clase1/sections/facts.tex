\section{Datos sobre \LaTeX}

\begin{frame}
	\frametitle{Datos sobre \LaTeX}

	\begin{columns}
		\begin{column}{.58\paperwidth}
			\begin{itemize}[<+->]
				\item

				      Leslie Barry Lamport en $1984$ documentó
				      en~\cite{Lamport1994} un conjunto de macros \TeX{} en
				      el \emph{Stanford Research Institute} con el propósito
				      de redistribuir a través de paquetes.
				      % Siguiendo el principio de desarrollo de software
				      % ``Don't repeat yourself''.

				      \

				\item

				      El mantenimiento del
				      \href{https://github.com/latex3/latex2e}{kernel \LaTeX{}2e}
				      está a cargo del
				      \href{https://www.latex-project.org/about/team}{Proyecto \LaTeX},
				      mientras que en
				      \href{https://ctan.org}{\emph{Comprehensive \TeX{} Archive Network}}
				      se encuentran los paquetes que la comunidad ha
				      publicado.

				      \

				\item

				      % tex-fmt: off
				      \textcolor{cyan}{\faIcon{globe}}~\href{https://tex.stackexchange.com}{Foro
					      de preguntas y respuestas de \LaTeX{}}.
				      % tex-fmt: on

				      \

				\item

				      El \href{https://tug.org}{\TeX{} User Group}
				      realiza talleres cada año sobre \LaTeX{} y afines.

				      \

				\item

				      Está licenciado como software libre bajo la \LaTeX{}
				      Project Public License.

				      \

				\item

				      % tex-fmt: off
				      \textcolor{cyan}{\faIcon{globe}}~\href{https://tex.stackexchange.com/q/339}{Lista
					      de editores de \LaTeX{}}.
				      % tex-fmt: on

				      \

				\item

				      Desde el 1 de noviembre del 2024~\cite{LaTeX40},
				      para documentos nuevos se recomienda emplear el
				      compilador Lua\TeX, pero el soporte para pdf\LaTeX{}
				      continuará.

				      \

				\item

				      % tex-fmt: off
				      \textcolor{red}{\faIcon{youtube}}~\href{https://www.youtube.com/watch?v=Pa3eQ_OlBeE}{Instalación
					      de una distribución de \LaTeX{} y del editor \TeX{}studio}.
				      % tex-fmt: on

				      \

				\item

				      Formateadores de sintáxis de \LaTeX{}:
				      \href{https://github.com/cmhughes/latexindent.pl}{latexindent}
				      y \href{https://github.com/WGUNDERWOOD/tex-fmt}{tex-fmt}.
			\end{itemize}
		\end{column}
		\begin{column}{.38\paperwidth}
			\begin{figure}[ht!]
				\centering
				\includegraphics[width=.32\paperwidth]{Lamport}
				\caption{Leslie Lamport, matemático del MIT.
					Creador de \LaTeX{} y ganador del premio Turing en el año 2013.}
			\end{figure}
		\end{column}
	\end{columns}
\end{frame}

\begin{frame}
	\frametitle{Alternativas a \LaTeX}

	\begin{itemize}[<+->]
		\item

		      \faIcon{github}~\href{https://github.com/typst/typst}{Typst}.
		      Un nuevo sistema de composición tipográfica basado en
		      marcado que está diseñado para ser tan potente como
		      \LaTeX{} y al mismo tiempo mucho más fácil de aprender y
		      usar.

		      \

		\item

		      \faIcon{github}~\href{https://github.com/texmacs/texmacs}{\TeX{}macs}.
		      Un procesador de textos científico y componente de
		      composición tipográfica del Proyecto GNU.
		      Se originó como una variante de GNU Emacs con
		      funcionalidades de \TeX, aunque no comparte código con
		      estos programas, aunque utiliza fuentes de \TeX.

		      \

		\item

		      \faIcon{globe}~\href{https://www.lyx.org}{LyX}.
		      Un procesador de documentos completo que fomenta un
		      enfoque de escritura basado en la estructura de sus documentos.

		      \

		\item

		      \textcolor{cyan}{\faIcon{globe}}~\href{https://wiki.contextgarden.net}{Con\TeX{}t}. % tex-fmt: skip
		      Un procesador de documentos de propósito general.
		      Al igual que \LaTeX, se deriva de \TeX.
		      Es especialmente adecuado para documentos
		      estructurados, producción automatizada de
		      documentos, tipografía de alta calidad y
		      composición tipográfica multilingüe.
		      Se basa parcialmente en el sistema de composición
		      tipográfica \TeX{} y utiliza un lenguaje de marcado
		      de documentos para la preparación de manuscritos.

		      \

		\item

		      \faIcon{globe}~\href{https://quarto.org}{Quarto}.
		      Un sistema de publicación científica y técnica de código
		      abierto.
		      Tiene integración con los lenguajes de programación:
		      Python, R\footnote{Vea más en~\url{https://quarto.org/docs/computations/r.html}.},
					Julia y Observable JavaScript.

		      \

		\item

		      \faIcon{globe}~\href{https://sile-typesetter.org}{SILE}.
		      Un sistema de composición tipográfica.
		      Su función es producir documentos impresos de alta calidad.
	\end{itemize}
\end{frame}


% https://texample.net/tex-workflow
\begin{frame}[t,fragile]
	\frametitle{Compiladores para \LaTeX}

	\begin{block}{Motores de \LaTeX}
		\begin{description}[<+->]
			\item[\href{https://tug.org/texlive/Contents/live/texmf-dist/doc/pdftex/manual/pdftex-a.pdf}{pdf\TeX}]

			      Genera directamente a PDF.
			      Introdujo mejoras en la composición tipográfica de
			      \TeX{}.
			      Fue desarrollado por Hàn Thế Thành y sus detalles de
			      implementación formaron la base de su tesis doctoral.


			\item[\href{https://xml.web.cern.ch/XML/lgc2/xetexmain.pdf}{Xe\TeX}]

			      Lee directamente archivos \TeX{} guardados en
			      codificación UTF-8.
			      Maneja la composición tipográfica multilingüe (por ejemplo, árabe) y
			      matemática basada en fuentes OpenType.

			\item[\href{https://mirrors.ibiblio.org/CTAN/systems/doc/luatex/luatex.pdf}{Lua\TeX}]

			      Es derivado de pdf\TeX{}, incorpora del lenguaje de
			      script Lua, que permite un control muy sofisticado del
			      motor \TeX{} a través de un lenguaje de script fácil de
			      usar.
		\end{description}
	\end{block}
	\pause
	\begin{center}
		\fbox{
			\alert{
				La compatibilidad entre los compiladores de \LaTeX{} y los
				paquetes puede variar.
			}
		}
	\end{center}

	\

	\begin{figure}[ht!]
		\centering
		\caption{Flujo de trabajo del compilador Lua\TeX.}
		\begin{tikzpicture}[node distance=3cm, auto,>=latex', thick]
			\path[use as bounding box] (-1,0) rectangle (10,-2);
			\path[->] node[format] (tex) {archivo \verb|.tex|};
			\path[->] node[format, right of=tex] (dvi) {archivo \verb|.dvi|} (tex) edge node {\TeX} (dvi);
			\path[->] node[format, right of=dvi] (ps) {archivo \verb|.ps|} node[medium, below of=dvi] (screen) {pantalla} (dvi) edge node {dvips} (ps) edge node[swap] {xdvi} (screen);
			\path[->] node[format, right of=ps] (pdf) {archivo \verb|.pdf|} node[medium, below of=ps] (print) {impresora} (ps) edge node {ps2pdf} (pdf)  edge node[swap] {gs} (screen)  edge (print);
			\path[->] (pdf) edge (screen) edge (print);
			\path[->, draw] (tex) -- (0,1) -| node[near start] {Lua\TeX} (pdf);
		\end{tikzpicture}
	\end{figure}
\end{frame}

\begin{frame}
  \frametitle{Reconozca en el teclado los caracteres más frecuentes en \LaTeX}

  \begin{figure}[ht!]
    \centering
    \includegraphics[width=.85\paperwidth]{keyboard}
    \caption{Para imprimir en el editor el carácter backslash
      \keystrokebftt{\bs} combine las teclas \texttt{alt} seguido de
      su posición.
      Para los caracteres de corchetes
      \keystrokebftt{[} \keystrokebftt{]}, anteponga la tecla
    \texttt{shift} y el signo de llave \keystrokebftt{\{} \keystrokebftt{\}}.}
  \end{figure}
\end{frame}

