\begin{frame}
	\frametitle{Estilos de \LaTeX}
	% https://ctan.org/topic/class
	% https://ctan.org/topic/cv
	\begin{center}
		\fbox{
			\alert{Sintáxis}:
			\mintinline{latex}|\documentclass{<nombre-de-la-clase>}|
		}
	\end{center}

	\pause

	\begin{block}{Clases estándar}
		\begin{itemize}
			\item

			      \mintinline{latex}|article|: Documentos
			      breves (con opciones: horizontal, a dos columnas, fleqn,
			      monócromo).

			\item

			      \mintinline{latex}|book|: Estilo libro (a doble
			      cara).

			\item

			      \mintinline{latex}|letter|: Estilo carta.


			\item

			      \mintinline{latex}|report|: Documentos
			      extensos + capítulos.

			\item

			      \mintinline{latex}|beamer|,
			      \mintinline{latex}|powerdot|,
			      \mintinline{latex}|ltx-talk|:
			      Estilo transparencias.

			\item

			      \mintinline{latex}|standalone|,
			      \mintinline{latex}|minimal|: Crea un PDF ajustado al
			      texto, también para pruebas mínimas.
		\end{itemize}
	\end{block}

	\pause

	\begin{block}{Clases extendidas}
		\vspace*{-.5\baselineskip}\setlength\belowdisplayshortskip{0pt}
		\begin{columns}
			\begin{column}{.52\paperwidth}
				\begin{itemize}
					\item

					      \mintinline{latex}|scrartcl|,
					      \mintinline{latex}|scrreprt|,
					      \mintinline{latex}|scrbook|,
					      \mintinline{latex}|scrlttr2|: KOMA-Script.

					\item

					      \mintinline{latex}|memoir|: Libro con diversas opciones.

					\item

					      \mintinline{latex}|powerdot|: Alternativa a
					      \mintinline{latex}|beamer|.
				\end{itemize}
			\end{column}
			\begin{column}{.44\paperwidth}
				\begin{itemize}
					\item

					      \mintinline{latex}|exam|: Creación de exámenes.

					\item

					      \mintinline{latex}|sciposter|:
					      Creación de póster.

					\item

					      \mintinline{latex}|paper|,
					      \mintinline{latex}|amsproc|,
					      \mintinline{latex}|amsbook|,
					      etc.
				\end{itemize}
			\end{column}
		\end{columns}
	\end{block}

	\pause

	\begin{block}{Clases de revistas de artículos científicos}
		\vspace*{-.5\baselineskip}\setlength\belowdisplayshortskip{0pt}
		\begin{columns}
			\begin{column}{.3\paperwidth}
				\begin{itemize}
					\item

					      \mintinline{latex}|amsart|

					\item

					      \mintinline{latex}|IEEEtran|


					\item

					      \mintinline{latex}|elsarticle|
				\end{itemize}
			\end{column}
			\begin{column}{.3\paperwidth}
				\begin{itemize}
					\item

					      \mintinline{latex}|acmart|

					\item

					      \mintinline{latex}|siamart250211|

					\item

					      \mintinline{latex}|sn-jnl|
				\end{itemize}
			\end{column}
			\begin{column}{.3\paperwidth}
				\begin{itemize}
					\item

					      \mintinline{latex}|aastex631|

					\item

					      \mintinline{latex}|revtex4|

					\item

					      \mintinline{latex}|abntex2|
				\end{itemize}
			\end{column}
		\end{columns}
	\end{block}

	% 	Niveles según la clase:
	% ◦ artículo: secciones
	% ◦ informe / libro: capítulos
	% • Herramientas de estructuración:

	% ◦ \tableofcontents : resumen
	% ◦ \index : índice
	% ◦ \footnote : nota de base de página
\end{frame}
