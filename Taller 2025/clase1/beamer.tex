\documentclass[8pt,aspectratio=1610,c,spanish]{beamer}
\usepackage[spanish]{babel}
\usefonttheme[onlymath]{serif}
\usetheme{Singapore}
\setbeamertemplate{frametitle}[default][center]
\setbeamertemplate{navigation symbols}{}
\let\oldforall\forall\renewcommand{\forall}{\oldforall \, }
\let\oldexist\exists\renewcommand{\exists}{\oldexist \: }

\begin{document}
\begin{frame}
\frametitle{Entornos Matemáticos}

\begin{definition}[Matriz simétrica]
Una matriz $A\in\mathbb{R}^{n\times n}$ es \alert{simétrica} si y solo si $\forall u,v\in\mathbb{R}^{n}$:
$\left\langle u,Av\right\rangle=\left\langle Au,v\right\rangle$.
\end{definition}

\begin{theorem}[Factorización $LDL^{T}$]
Si $A\in\mathbb{R}^{n\times n}$ es simétrica y $\forall k\in\left\{1,\dotsc,n-1\right\}$: $A\left[1:k, 1:k\right]$
es no singular, entonces $\exists L\in\mathbb{R}^{n\times n}$ triangular inferior de unos en su diagonal y
$D=\operatorname{diag}\left(d_{1},\dotsc,d_{n}\right)$ tal que $A=LDL^{T}$. La factorización es única.
\end{theorem}

\begin{proof}Queda como ejercicio para el lector.\end{proof}

\begin{corollary}[Factorización de Cholesky]
Si $A\in\mathbb{R}^{n\times n}$ es simétrica definida positiva, entonces $\exists G\in\mathbb{R}^{n\times n}$ tal que
$A=GG^{T}$ y las entradas de su diagonal son positivas.
\end{corollary}

\begin{lemma}\end{lemma}

\begin{columns}
\begin{column}{.48\paperwidth}
	\begin{example}[Ejemplo de Aplicación]\end{example}
\end{column}
\begin{column}{.48\paperwidth}
	\begin{examples}[Dos contraejemplos]\end{examples}
\end{column}
\end{columns}
\end{frame}
\end{document}
