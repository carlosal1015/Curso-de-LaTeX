\documentclass{report}
\usepackage{amsmath}
\usepackage{amssymb}

\title{Teoría de Funciones Complejas}
\author{Grupo Estudiantil de Matemática}
\date{6 de septiembre de 2025}

\begin{document}
\part{Elementos de Teoría de Funciones}
\chapter{Números Complejos y Funciones Continuas}
\section{El campo $\mathbb{C}$ de los números complejos}
\subsection{El campo $\mathbb{C}$}
\subsection{Producto interno y valor absoluto}
\subsection{Mapeos que preservan el ángulo}
\section{Conceptos Topológicos Fundamentales}
\subsection{Espacios métricos}
\subsection{Conjuntos abiertos y cerrados}
\subsection{Sucesiones convergentes}
\section{Sucesiones convergentes de números complejos}
\section{Series convergentes y absolutamente convergentes}
\section{Funciones Continuas}
\chapter{Cálculo diferencial complejo}
\section{Ecuaciones de Cauchy-Riemann}
\section{Funciones holomórficas}
\section{Derivada parcial respecto a $z$ y $\overline{z}$}

\part{Teoría de Cauchy}
\chapter{Cálculo integral complejo}
\section{Integración sobre intervalos de $\mathbb{R}$}
\section{Integral de caminos en $\mathbb{C}$}
\section{Propiedades de integrales de camino complejas}
\chapter{El Teorema Integral de Cauchy}
\chapter{La fórmula integral de Cauchy para discos}
\chapter{Series de Taylor Especiales}

\part{Teoría de Funciones de Cauchy-Weierstrass}
\chapter{Teorema de la identidad  y de Liouville}
\chapter{Teoremas de Convergencia de Weierstrass}
\section{Funciones enteras y el principio del máximo}
\chapter{Teorema Fundamental del Álgebra}
\chapter{Grupos de $\operatorname{Aut}\left(E\right)$}
\end{document}
