\section{Matemáticas en \LaTeX}

\begin{frame}
	\frametitle{\secname}

	\begin{block}{\AmS-\LaTeX}
		\begin{columns}
			\begin{column}{.65\paperwidth}
				\LaTeX{} básico ofrece excelentes capacidades de composición
				matemática para documentos sencillos.
				Sin embargo, cuando se utilizan con frecuencia ecuaciones complejas
				o construcciones matemáticas más avanzadas, se necesita algo más.

				Si bien es posible definir nuevos comandos o entornos para
				simplificar la escritura de fórmulas, esta no es la mejor solución.
				A principios de los noventa, la
				\alert{American Mathematical Society (AMS)} proporcionó un
				paquete importante: \mintinline{latex}|amsmath|, que
				simplificó considerablemente la preparación de documentos
				matemáticos.
			\end{column}
			\begin{column}{.25\paperwidth}
				\begin{figure}[ht!]
					\centering
					\includesvg[width=.25\paperwidth]{ams}
					\caption{Aprenda más sobre \AmS-\LaTeX{} en~\cite{AMS2025}.}
				\end{figure}
			\end{column}
		\end{columns}
	\end{block}

	\

	Al final de esta sesión, podrás crear un documento básico con
	ecuaciones matemáticas basado en los siguientes paquetes.
	\begin{block}{Paquetes de la \AmS-\LaTeX}
		\vspace*{-\baselineskip}\setlength\belowdisplayshortskip{0pt}
		\begin{columns}
			\begin{column}{.2\paperwidth}
				\begin{itemize}
					\item

					      \mintinline{latex}|asmath|.
				\end{itemize}
			\end{column}
			\begin{column}{.2\paperwidth}
				\begin{itemize}
					\item

					      \mintinline{latex}|amssymb|.
				\end{itemize}
			\end{column}
			\begin{column}{.2\paperwidth}
				\begin{itemize}
					\item

					      \mintinline{latex}|amsthm|.
				\end{itemize}
			\end{column}
			\begin{column}{.2\paperwidth}
				\begin{itemize}
					\item

					      \mintinline{latex}|mathtools|.
				\end{itemize}
			\end{column}
		\end{columns}
	\end{block}

	\begin{block}{Paquetes de extensión}
		\vspace*{-\baselineskip}\setlength\belowdisplayshortskip{0pt}
		\begin{columns}
			\begin{column}{.2\paperwidth}
				\begin{itemize}
					\item

					      \mintinline{latex}|cancel|.

					\item

					      \mintinline{latex}|euler|.
				\end{itemize}
			\end{column}
			\begin{column}{.2\paperwidth}
				\begin{itemize}
					\item

					      \mintinline{latex}|fouriernc|.

					\item

					      \mintinline{latex}|siunitx|.
				\end{itemize}
			\end{column}
			\begin{column}{.2\paperwidth}
				\begin{itemize}
					\item

					      \mintinline{latex}|optidef|.

					\item

					      \mintinline{latex}|tikz-cd|.
				\end{itemize}
			\end{column}
			\begin{column}{.2\paperwidth}
				\begin{itemize}
					\item

					      \mintinline{latex}|unicode-math|.

					\item

					      \mintinline{latex}|diffcoeff|.
				\end{itemize}
			\end{column}
		\end{columns}
	\end{block}
\end{frame}

\begin{frame}
	\frametitle{\secname}

	Las dos maneras más simples de invocar el modo matemático son:

	\begin{table}[ht!]
		\centering
		\begin{tabular}{cc}
			\mintinline{latex}|$...$|                                & expresiones matemáticas no desplegadas          \\
			\mintinline{latex}|\begin{math}...\end{math}|            & expresiones matemáticas no desplegadas          \\
			\mintinline{latex}|\[...\]|                              & expresiones matemáticas desplegadas y centradas \\
			\mintinline{latex}|\begin{displaymath}\end{displaymath}| & expresiones matemáticas desplegadas y centradas \\
		\end{tabular}
	\end{table}

	\begin{equation*}
		\forall\varepsilon>0:\exists\delta>0\text{ tal que }
		\forall x\in\operatorname{dom}\left(f\right):
		\left|x-a\right|<\delta\implies\left|f\left(a\right)-f\left(x\right)\right|<\varepsilon.
	\end{equation*}

	\begin{columns}
		\begin{column}{.54\paperwidth}
			\begin{table}[ht!]
				\centering
				\begin{tabular}{|cl|cl|cl|}
					\hline
					$\alpha$      & \mintinline{latex}|\alpha|      & $\mu$     & \mintinline{latex}|\mu|     & $\varsigma$ & \mintinline{latex}|\varsigma| \\
					\hline
					$\beta$       & \mintinline{latex}|\beta|       & $\nu$     & \mintinline{latex}|\nu|     & $\lambda$   & \mintinline{latex}|\lambda|   \\
					\hline
					$\gamma$      & \mintinline{latex}|\gamma|      & $\psi$    & \mintinline{latex}|\psi|    & $\tau$      & \mintinline{latex}|\tau|      \\
					\hline
					$\delta$      & \mintinline{latex}|\delta|      & $\phi$    & \mintinline{latex}|\phi|    & $\theta$    & \mintinline{latex}|\theta|    \\
					\hline
					$\epsilon$    & \mintinline{latex}|\epsilon|    & $\varphi$ & \mintinline{latex}|\varphi| & $\vartheta$ & \mintinline{latex}|\vartheta| \\
					\hline
					$\varepsilon$ & \mintinline{latex}|\varepsilon| & $\pi$     & \mintinline{latex}|\pi|     & $\upsilon$  & \mintinline{latex}|\upsilon|  \\
					\hline
					$\eta$        & \mintinline{latex}|\eta|        & $\varpi$  & \mintinline{latex}|\varpi|  & $\chi$      & \mintinline{latex}|\chi|      \\
					\hline
					$\iota$       & \mintinline{latex}|\iota|       & $\rho$    & \mintinline{latex}|\rho|    & $\xi$       & \mintinline{latex}|\xi|       \\
					\hline
					$\kappa$      & \mintinline{latex}|\kappa|      & $\varrho$ & \mintinline{latex}|\varrho| & $\zeta$     & \mintinline{latex}|\zeta|     \\
					\hline
					$\varkappa$   & \mintinline{latex}|\varkappa|   & $\sigma$  & \mintinline{latex}|\sigma|  & $\omega$    & \mintinline{latex}|\omega|    \\
					\hline
				\end{tabular}
			\end{table}
		\end{column}
		\begin{column}{.36\paperwidth}
			\begin{table}[ht!]
				\centering
				\begin{tabular}{|cl|cl|}
					\hline
					$\Delta$   & \mintinline{latex}|\Delta|   & $\varDelta$   & \mintinline{latex}|\varDelta|   \\
					\hline
					$\Lambda$  & \mintinline{latex}|\Lambda|  & $\varLambda$  & \mintinline{latex}|\varLambda|  \\
					\hline
					$\Pi$      & \mintinline{latex}|\Pi|      & $\varPi$      & \mintinline{latex}|\varPi|      \\
					\hline
					$\Psi$     & \mintinline{latex}|\Psi|     & $\varPsi$     & \mintinline{latex}|\varPsi|     \\
					\hline
					$\Phi$     & \mintinline{latex}|\Phi|     & $\varPhi$     & \mintinline{latex}|\varPhi|     \\
					\hline
					$\Sigma$   & \mintinline{latex}|\Sigma|   & $\varSigma$   & \mintinline{latex}|\varSigma|   \\
					\hline
					$\Theta$   & \mintinline{latex}|\Theta|   & $\varTheta$   & \mintinline{latex}|\varTheta|   \\
					\hline
					$\Upsilon$ & \mintinline{latex}|\Upsilon| & $\varUpsilon$ & \mintinline{latex}|\varUpsilon| \\
					\hline
					$\Xi$      & \mintinline{latex}|\Xi|      & $\varXi$      & \mintinline{latex}|\varXi|      \\
					\hline
					$\Omega$   & \mintinline{latex}|\Omega|   & $\varOmega$   & \mintinline{latex}|\varOmega|   \\
					\hline
					$\Gamma$   & \mintinline{latex}|\Gamma|   & $\varGamma$   & \mintinline{latex}|\varGamma|   \\
					\hline
					$\digamma$ & \mintinline{latex}|\digamma| &               &                                 \\
					\hline
				\end{tabular}
			\end{table}
		\end{column}
	\end{columns}
\end{frame}

\begin{frame}
	\frametitle{\secname}

	$\sum_{n=1}^{\infty}a_{n}$

	$\sum\limits_{n=1}^{\infty}a_{n}$

	\[\sum_{n=1}^{\infty}a_{n}\]

	\[\sum\nolimits_{n=1}^{\infty}a_{n}\]

	$\smallint_{a}^{b}$

	$\int_{a}^{b}f$

	$\int_{[a,b]}f$

	\[\int_{[a,b]}f\]

	$\lim_{x\to0}\frac{\sen x}{x}=1$.
\end{frame}
