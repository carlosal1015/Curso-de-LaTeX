% \section{\protect{\mintinline{latex}|amsmath|}}

\begin{frame}[fragile]
	\frametitle{\secname}

	\begin{equation}
		\left(a+b\right)^{2}=
		a^{2}+2ab+b^{2}.
	\end{equation}
	\begin{displaymath}
		\sen^{2}\eta+\cos^{2}\eta=1.
	\end{displaymath}

	\begin{align}
		x^{2}+y^{2} & = z^{2}. \\
		x^{3}+y^{3} & <z^{3}
	\end{align}

	\begin{eqnarray}
		x^{2}+y^{2} & =& z^{2}. \\
		x^{3}+y^{3} & <&z^{3}
	\end{eqnarray}

	\begin{equation*}
		n^{p}+m^{p}\neq k^{p}\qquad p>2.
	\end{equation*}
\end{frame}

% \section{\mintinline{latex}|amssymb|}

% \section{\mintinline{latex}|amsthm|}

% \section{\mintinline{latex}|mathtools|}

% \section{\mintinline{latex}|diffcoeff|}

% \section{\mintinline{latex}|optidef|}

\begin{frame}[fragile]
	\frametitle{\secname}

	\begin{multline}
		\text{Primera línea de una línea múltiple}
		\\
		\text{Línea media centrada}
		\\
		\shoveright{\text{Una línea media derecha}} \\
		\text{Otra línea media centrada}
		\\
		\text{Otra línea media centrada} \\
		\shoveleft{\text{Una línea media izquierda}}
		\\
		\text{Última línea de la línea múltiple}
	\end{multline}

	\begin{multline}
		\tag{2}
		\sum_{t\in\mathbf{T}}
		\int_{a}^{t}
		\biggl\lbrace
		\int_{a}^{t}
		f\left(t-x\right)^{2}\,
		g\left(y\right)^{2}\,dx
		\biggr\rbrace
		\,dy \\
		= \sum_{t\notin\mathbf{T}}
		\int_{t}^{a}
		\biggl\lbrace
		g\left(y\right)^{2}
		\int_{t}^{a}
		f\left(x\right)^{2}\,dx
		\biggr\rbrace\,dy
	\end{multline}
	\setlength\multlinegap{0pt}
	\begin{multline}
		\tag{2}
		\sum_{t\in\mathbf{T}}
		\int_{a}^{t}
		\biggl\lbrace
		\int_{a}^{t}
		f\left(t-x\right)^{2}\,
		g\left(y\right)^{2}\,dx
		\biggr\rbrace\,dy \\
		=\sum_{t\notin\mathbf{T}}
		\int_{t}^{a}
		\biggl\lbrace
		g\left(y\right)^{2}
		\int_{t}^{a}
		f\left(x\right)^{2}\,dx
		\biggr\rbrace
		\,dy
	\end{multline}
\end{frame}

\begin{frame}[fragile]
	\frametitle{\secname}

	\begin{equation}
		\begin{split}
			\left(a+b\right)^{4} & = \left(a+b\right)^{2}\left(a+b\right)^{2}.                 \\
			                     & = \left(a^{2}+2ab+b^{2}\right)\left(a^{2}+2ab+b^{2}\right). \\
			                     & = a^{4}+4a^{3}b+6a^{2}b^{2}+4ab^{3}+b^{4}.
		\end{split}
	\end{equation}

	\begin{equation}
		\begin{split}
			\left(a+b\right)^{3} & = \left(a+b\right)\left(a+b\right)^{2}.         \\
			                     & = \left(a+b\right)\left(a^{2}+2ab+b^{2}\right). \\
			                     & = a^{3}+3a^{2}b+3ab^{2}+b^{3}.
		\end{split}
	\end{equation}

	\begin{equation}
		\begin{split}
			f_{h,\ve}\left(x,y\right) & = \ve\bfE_{x,y}
			\int_{0}^{\tve}
			L_{x,\yvf(\ve u)}\vf(x)\,du                                          \\
			                          & = h\int L_{x,z}\vf(x)\rho_{x}(dz)        \\
			                          & \relphantom{=} {} + h\biggl[
			\frac{1}{\tve}
			\biggl(
			\bfE_{y}\int_{0}^{\tve}
			L_{x, y^x(s)} \vf(x) \,ds
			- \tve \int L_{x, z} \vf(x) \rho_x(dz)
			\biggr) +                                                            \\
			                          & \relphantom{=} \phantom{{} + h \biggl[ }
				\frac{1}{\tve}
				\biggl( \bfE_{y} \int_0^{\tve}
				L_{x, y^x(s)} \vf(x) \,ds
				- \bfE_{x, y} \int_0^{\tve} L_{x, \yvf(\ve s)}
				\vf(x) \,ds
				\biggr) \biggr]
		\end{split}
	\end{equation}
\end{frame}

\begin{frame}[fragile]
	\frametitle{\secname}

	\begin{gather}
		\left(a+b\right)^{2}=a^{2}+2ab+b^{2}. \\
		\left(a+b\right)\cdot\left(a-b\right)=
		a^{2}-b^{2}.
	\end{gather}

	\begin{gather}
		D\left(a,r\right)\equiv
		\left\{z\in\mathbf{C}\colon \left|z-a\right|<r\right\}. \notag \\
		\operatorname{seg}\left(a,r\right)\equiv
		\left\{
		z\in\mathbf{C}\colon
		\Im z<\Im a,\ \left|z-a\right|< r \right\}. \\
		C\left(E,\theta,r\right)\equiv
		\bigcup_{e \in E}
		c\left(e,\theta,r\right).
	\end{gather}
\end{frame}

\begin{frame}[fragile]
	\frametitle{\secname}

	\begin{align}
		\left(a+b\right)^{3} & = \left(a+b\right)\left(a+b\right)^{2}.         \\
		                     & = \left(a+b\right)\left(a^{2}+2ab+b^{2}\right). \\
		                     & = a^{3}+3a^{2}b+3ab^{2}+b^{3}.
	\end{align}
	\begin{align}
		x^{2} + y^{2} & = 1             \\
		x             & = \sqrt{1-y^2}.
	\end{align}
\end{frame}

\begin{frame}[fragile]
	\frametitle{\secname}

	This example has two column-pairs.
	\begin{align}
		\text{Compare }
		x^{2}+y^{2} & = 1              & x^{3} + y^{3} & = 1                 \\
		x           & = \sqrt{1-y^{2}} & x             & = \sqrt[3]{1-y^{3}}
	\end{align}
	This example has three column-pairs.
	\begin{align}
		x              & = y              & X              & = Y              & a           & = b+c         \\
		x + x^{\prime} & = y + y^{\prime} & X + X^{\prime} & = Y + Y^{\prime} & a^{\prime}b & = c^{\prime}b
	\end{align}
\end{frame}

\begin{frame}[fragile]
	\frametitle{\secname}

	This example has two column-pairs.
	\begin{flalign}
		\text{Compare }
		x^{2} + y^{2} & = 1              & x^{3} + y^{3} & = 1                 \\
		x             & = \sqrt{1-y^{2}} & x             & = \sqrt[3]{1-y^{3}}
	\end{flalign}
	This example has three column-pairs.
	\begin{flalign}
		x              & = y              & X              & = Y              & a           & = b+c         \\
		x + x^{\prime} & = y + y^{\prime} & X + X^{\prime} & = Y + Y^{\prime} & a^{\prime}b & = c^{\prime}b
	\end{flalign}

	This example has two column-pairs.
	\renewcommand\minalignsep{0pt}
	\begin{align}
		\text{Compare }
		x^{2} + y^{2} & = 1              & x^{3} + y^{3} & = 1                 \\
		x             & = \sqrt{1-y^{2}} & x             & = \sqrt[3]{1-y^{3}}
	\end{align}
	This example has three column-pairs.
	\renewcommand\minalignsep{15pt}
	\begin{flalign}
		x      & = y      & X      & = Y      & a   & = b+c \\
		x + x’ & = y + y’ & X + X’ & = Y + Y’ & a’b & = c’b
	\end{flalign}
\end{frame}

\begin{frame}[fragile]
	\frametitle{\secname: \mintinline{latex}|mathtools|}

	\begin{itemize}
		\item

		      \mintinline{latex}|[sumlimits]|

		      $\sum_{j=1}^{n}$

		\item

		      \mintinline{latex}|[intlimits]|


		      $\int_{a}^{b}$

		\item

		      \mintinline{latex}|[leqno]|

		      \begin{equation}
			      e^{i\pi}+1=0.
		      \end{equation}
	\end{itemize}
	% tex-fmt: off
\begin{exampletwoup}
%\usepackage{amssymb} %\usepackage{mathrsfs}
\begin{equation}\mathbb{ABCDEFGH}\end{equation}
\begin{equation*}\mathscr{ABCDEFGH}\end{equation*}
\[\mathcal{ABCDEFGH}\mathfrak{ABCDEFGH}\]

\begin{math}
\forall j\in\left\{1,\dotsc,n\right\},
x_{j}\in\mathbb{C}:
\left|\sum_{j=1}^{n}x_{j}\right|\leq
\sum_{j=1}^{n}\left|x_{j}\right|
\end{math}.

$\alpha+\beta+\gamma+\tau+\theta+\phi+\omega=\pi$.
\end{exampletwoup}
% tex-fmt: on

\end{frame}

\begin{frame}[fragile]
	\frametitle{\secname}

	% https://arxiv.org/pdf/1402.4982
	% https://www.ucm.es/data/cont/docs/1346-2019-04-12-BaSix%20LaTeX%20ba%CC%81sico%20con%20ejercicios%20resueltos%20-%20Noir16.pdf
	% https://br.mirrors.cicku.me/ctan/macros/latex/contrib/microtype/microtype.pdf
	\[
		\int_{-1}^{1}
		f\left(t\right)
		\dl{\alpha\left(t\right)}\approx
		Af\left(-\frac{\sqrt{3}}{3}\right)+
		Bf\left(\frac{\sqrt{3}}{3}\right).
	\]
\end{frame}

\begin{frame}[fragile]
	\frametitle{\secname: \mintinline{latex}|amssymb|}

\end{frame}

\begin{frame}[fragile]
	\frametitle{\secname: \mintinline{latex}|amsthm|}

	\begin{example}% tex-fmt: off
\begin{exampletwoup}
\begin{theorem}[Hahn-Banach]
Sean $\left(X,\mathbb{C}\right)$ un espacio vectorial y $M$ un
subespacio vectorial de $X$.
Si $p\colon X\to\mathbb{R}$ es una seminorma y
$f\colon M\to\mathbb{C}$ una funcional lineal que satisface
$\forall m\in M:\left|f\left(m\right)\right|\leq p\left(m\right)$,
entonces $\exists F\colon X\to\mathbb{C}$ tal que
\begin{columns}
\begin{column}{.4\paperwidth}
\begin{itemize}
\item

$\forall m\in M: F\left(m\right)=f\left(m\right)$.
\end{itemize}
\end{column}
\begin{column}{.4\paperwidth}
\begin{itemize}
\item

$\forall x\in X: \left|F\left(x\right)\right|\leq p\left(x\right)$.
\end{itemize}
\end{column}
\end{columns}
\end{theorem}
\end{exampletwoup}
% tex-fmt: on
\end{example}
	% \begin{example}% tex-fmt: off
\begin{exampletwoup}
	\begin{definition}[Derivada de Clarke generalizada]
		Sea $f\colon\mathbb{R}^{n}\to\mathbb{R}$ una función localmente
		lipschitz, la derivada de Clarke generalizada de $f$ en
		$x\in\mathbb{R}^{n}$ en la dirección $v\in\mathbb{R}^{n}$
		% \begin{align*}
		% 	0
		% 	%f^{\circ}\left(x,v\right)       %& \coloneqq
		% 	% \limsup_{y\rightarrow x,h\downarrow0}\frac{f\left(y+hv\right)-f\left(y\right)}{h}.
		% 	% \shortintertext{El gradiente de Clark generalizado de $f$ en $x$}
		% 	% \partial^{\circ}f\left(x\right) & \coloneqq
		% 	% \left\{
		% 	% \xi\in\mathbb{R}^{n}\mid
		% 	% \left\langle\xi,v\right\rangle\leq
		% 	% f^{\circ}\left(x,v\right),
		% 	% \forall v\in\mathbb{R}^{n}
		% 	% \right\}.
		% \end{align*}
	\end{definition}
\end{exampletwoup}
% tex-fmt: on
\end{example}
\end{frame}

\begin{frame}[fragile]
	\frametitle{\secname: \mintinline{latex}|optidef|}
	\begin{mini}<b>
		{w}{f(w)+R(w+6x)}
		{\label{eq:Example1}}{}
		\addConstraint{p(w)}{=0}
		\addConstraint{q(w)}{=0}
		\addConstraint{r(w)}{=0\labelOP{testLabel}}
		\addConstraint{n(w)}{= 6}
		\addConstraint{L(w)+r(x)}{=Kw+p}
		\addConstraint{h(x)}{=0.}
	\end{mini}
\end{frame}

\begin{frame}[fragile]
	\frametitle{\secname: \mintinline{latex}|diffcoeff|}

	\begin{equation}
		\forall z\in\mathbb{C}:\Gamma\left(z\right)\coloneqq
		\int_{0}^{\infty}t^{z-1}e^{-t}\dl t.
	\end{equation}

	\begin{equation*}
		\diffp{u}{t}+\diff{u}{x}+\diff[2]{u}{x}+u=0.
	\end{equation*}

\end{frame}

% \section{Paquete \protect{\mintinline{latex}|nicefrac|}}
